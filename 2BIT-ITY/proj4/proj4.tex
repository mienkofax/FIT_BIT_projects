%
%Projekt:	ITY 4.projekt 
%Autor:		Peter Tisovcik - xtisov00@stud.fit.vutbr.cz
%Datum:		19.04.2015
%
\documentclass[11pt, a4paper, titlepage] {article} 
    \usepackage[left=2cm, text={17cm, 24cm}, top=3cm ]{geometry}  
    \usepackage[czech]{babel}
	\usepackage[utf8]{inputenc}
	\usepackage{url}
	\usepackage{times} 
	\usepackage[T1]{fontenc}
	\pagestyle{plain}  

	\bibliographystyle{czplain}
	
	\newcommand{\myuv}[1]{\quotedblbase #1\textquotedblleft}
	
\begin{document}

%%%%% Title page 
\begin{titlepage}
\begin{center}
	{\LARGE\textsc{Vysoké učení technické v~Brně}}\\
\medskip
{\Large\textsc{Fakulta informačních technologií}}\\
\vspace{\stretch{0.382}}
{\LARGE Typografie a~publikování\,--\,4.\,projekt}\\
\medskip
{\Huge Bibliografické citace}\\
\vspace{\stretch{0.618}}
\end{center}
{\Large \today \hfill Peter Tisovčík}
\end{titlepage}

\section{Čo je to typografia}
Typografia sa zaoberá grafickým návrhom respektíve prácou s písmom a textom. Typografia najviac ovyplyvňuje vzhľad výrobkov a stavieb. Stretávame sa s ňou každý deň a ani si to neuvedomujeme. Základným prvkom je písmo, pomocou, ktorého sa uchovávajú a prenášajú informácie \cite{Sirutek:Pravidla}.

Z rozvojom komunikačných schopností ľudí a spoločností sa zvyšuje počet dokumentov rôzneho obsahu, formy a kvality \cite{Bedrichova:vyuzitie}.

Najväčšie odvetvie, ktoré sa zaoberá typografiou je odvetvie marketingu, kde sa vytvárajú tiskoviny, reklamné predmety a obaly pre produkty. Obaly predávajú produkt a čím kvalitnejšie je spracovaný tým má produkt väčší úspech \cite{Typografia:Forma}.

Formátovanie a obsah textu nie je všetko, čo spadá do typografie. Do typografie taktiež patrí aj tlač. Je to neodmysliteľná súčasť tohto odvetvia. Správna tlač je dôležitá predovšetkým, pre to aby sa vierohodne zobrazil vysádzaný text na pripravený materiál \cite{Typografia:Tlac}.

\section{Typografia a \LaTeX }
Štýl, ktorým sa tvoria dokumenty v tomto systému je odlišný od grafických editorov. Zo začiatku to môže odradiť väčšinu používateľov, kôli zložitosti a pracnému písaniu jednotlivých príkazov. K tomu môže ľahko dopomôcť kvalitná kniha \cite{Rybicka:Latex}. Ľahko dostupné materiály sú aj v anglickom jazyku, ktoré sú stručné a ponúkajú základnú ale aj pokročilú sadzbu dokumentov v \LaTeX u \cite{Wikipedia:Latex} \cite{Kopka:Latech}. Netreba zabúdať, že \LaTeX  ponúka mnoho možností, ktoré sa v ňom dajú vytvoriť. Užívateľ si môže pridať nové písmo, vytvoriť jednoduchú alebo zložitú 3D grafiku v PDF dokumentoch bez použitia iných nástrojov \cite{Wagner:Zpravodaj}.

\section{Typografia vo webdizajne}
\emph{Web Design is 95\% Typography} \cite{Reichestein:Typografia}. Dôležitejšie ako vzhľad webovej prezentácie je jej obsah a štýl akým je obsah podaný používateľovi. Stránky sa čoraz viacej podobajú vzhľadom a aj obsahom, preto je potrebné sa odlíšiť od ostatných a mať vytvorené texty štylisticky a gramaticky na vysokej úrovni. Dôležitým aspeknom je čitateľnosť písma a vizuálne oddelenie častí, ktoré sú viac, či menej dôležité. Netreba však zabúdať na zarovnania do blokov alebo riadkovanie \cite{Justina:Pismo}.



\newpage
\bibliography{citacie}

\end{document}