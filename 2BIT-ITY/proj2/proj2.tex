%
%Projekt:	ITY 2.projekt 
%Autor:		Peter Tisovcik - xtisov00@stud.fit.vutbr.cz
%Datum:		11.3.2015
%
\documentclass[11pt, a4paper, twocolumn, titlepage] {article} 
    \usepackage[left=1.5cm, text={18cm, 25cm}, top=2.5cm ]{geometry}  
    \usepackage[czech]{babel}
	\usepackage[utf8]{inputenc}
	\usepackage{times} 
	\usepackage[IL2]{fontenc} 
	\usepackage{amsmath,amsthm,amssymb}
	\usepackage{mdwlist}

	\theoremstyle{definition}
	\newtheorem{definition}{Definice}[section]
	\newtheorem{algorithm}[definition]{Algoritmus}
	\newtheorem{veta}{Věta}
	
	
\begin{document}

%%%%% Title page 
\begin{titlepage}
\begin{center}
	\Huge
	\textsc{Fakulta informačních technologií\\Vysoké učení technické v Brně}\\
	\vspace{\stretch{0.382}}
	\Large Typografie a~publikování\,\textendash\,2. projekt\\
	Sazba dokumentů a~matematických výrazů
	\vspace{\stretch{0.618}}
\end{center}
{\LARGE 2015 \hfill Peter Tisovčík}
\end{titlepage}
	

%%%%% Druha strana 
\section*{Úvod}
V~této úloze si vyzkoušíme sazbu titulní strany, matematických vzorců, prostředí a~dalších textových struktur obvyklých pro technicky zaměřené texty (například rovnice (\ref{rovnica-1}) nebo definice \ref{def:matematicky_text-def1-1} na straně \pageref{def:matematicky_text-def1-1}).

Na titulní straně je využito sázení nadpisu podle optického středu s~využitím zlatého řezu. Tento postup byl probírán na přednášce.

\section{Matematický text}
Nejprve se podíváme na sázení matematických symbolů a~výrazů v~plynulém textu. Pro množinu $V$ označuje $\mathrm{card}(V)$ kardinalitu $V$.
Pro množinu $V$ reprezentuje $V^*$ volný monoid generovaný množinou $V$ s~operací konkatenace.
Prvek identity ve volném monoidu $V^*$ značíme symbolem $\varepsilon$.
Nechť $V^+ = V^* - \{\varepsilon\}$. Algebraicky je tedy $V^+$ volná pologrupa generovaná množinou $V$ s~operací konkatenace.
Konečnou neprázdnou množinu $V$ nazvěme $abeceda$.
Pro $\omega \in V^*$ označuje $|w|$ délku řetězce $w$. Pro $W \subseteq V$ označuje $\mathrm{occur}(w,W)$ počet výskytů symbolů z~$W$ v~řetězci $w$ a~sym$(w, i)$ určuje $i$-tý symbol řetězce $w$; například $\mathrm{sym}(abcd,3) = c$.

Nyní zkusíme sazbu definic a~vět s~využitím balíku \verb|amsthm|.

\begin{definition} \label{def:matematicky_text-def1-1}
\emph{Bezkontextová gramatika} je čtveřice $G=(V,T,P,S)$, kde $V$ je totální abeceda, $T \subseteq V$ je abeceda terminálů, $S \in (V-T)$ je startující symbol a~$P$ je konečná množina \emph{pravidel} tvaru $q\colon A\rightarrow \alpha$, kde $A\in (V-T)$, $\alpha \in V^*$ a $q$ je návěští tohoto pravidla. Nechť $N = V - T$ značí abecedu neterminálů. 
Pokud $q\colon A \rightarrow \alpha\in P$, $\gamma$, $\delta \in V^*$, $G$ provádí derivační krok z~$\gamma A \delta$ do $\gamma \alpha \delta$ podle pravidla $q\colon A \rightarrow \alpha$, symbolicky píšeme $\gamma A \delta \Rightarrow \gamma\alpha\delta \ [ q\colon A \rightarrow \alpha ]$ nebo zjednodušeně $\gamma A \delta \Rightarrow \gamma\alpha\delta$. Standardním způsobem definujeme $\Rightarrow^m$, kde $m \geq 0$. Dále definujeme tranzitivní uzávěr $\Rightarrow^+$ a~tranzitivně-reflexivní uzávěr $\Rightarrow^*$.
\end{definition}


Algoritmus můžeme uvádět podobně jako definice textově, nebo využít pseudokódu vysázeného ve vhodném prostředí (například \verb|algorithm2e|).

\begin{algorithm}
\emph{Algoritmus pro ověření bezkontextovosti gramatiky. Mějme gramatiku $G = (N, T, P, S)$.}
\begin{enumerate}
\item \label{alg-1} \emph{Pro každé pravidlo $p \in P$ proveď test, zda $p$ na levé straně obsahuje právě jeden symbol z~$N$}
\item \emph {Pokud všechna pravidla splňují podmínku z~kroku \ref{alg-1}, tak je gramatika $G$ bezkontextová.}
\end{enumerate}
\end{algorithm}

\begin{definition}
\emph{Jazyk} definovaný gramatikou $G$ definujeme jako $L(G) = \{w \in T^*|S \Rightarrow^*w\}$.
\end{definition}


\subsection{Podsekce obsahující větu}
\begin{definition}
Necht $L$ je libovolný jazyk. $L$ je \emph{bezkontextový jazyk}, když a~jen když $L = L(G)$, kde $G$ je libovolná bezkontextová gramatika.
\end{definition}

\begin{definition}
Množinu $\mathcal{L}_{CF} = \{L|L$ je bezkontextový jazyk$\}$ nazýváme \emph{třídou bezkontextových jazyků}.
\end{definition}

\begin{veta} \label{veta-1}
Nechť $L_{abc} = \{ a^nb^nc^n|n \geq 0 \}$. Platí, že $L_{abc} \notin \mathcal{L}_{CF}$.
\end{veta}

\begin{proof}
Důkaz se provede pomocí Pumping lemma pro bezkontextové jazyky, kdy ukážeme, že není možné, aby platilo, což bude implikovat pravdivost věty \ref{veta-1}.
\end{proof}

\section{Rovnice a odkazy}
Složitější matematické formulace sázíme mimo plynulý text. Lze umístit několik výrazů na jeden řádek, ale pak je třeba tyto vhodně oddělit, například příkazem \verb|\quad|.

$$\sqrt[x^2]{y^3_0} \quad \mathbb{N} = \{0,1,2,...\} \quad x^{y^y} \neq x^{yy} \quad z_{i_j} \neq z_{ij}$$

V rovnici (\ref{rovnica-1}) jsou využity tři typy závorek s~různou explicitně definovanou velikostí.
\begin{eqnarray} 
\bigg{\{} \Big[ \big(a + b \big) * c \Big]^d + 1\bigg{\}} &= &x \label{rovnica-1}\\
\lim_{x\to\infty}\frac{\sin^2 x + \cos^2 x}{4}  &= &y \nonumber
\end{eqnarray}

V této větě vidíme, jak vypadá implicitní vysázení limity $\lim_{x\to\infty}f(n)$ v~normálním odstavci textu. Podobně je to i~s~dalšími symboly jako $\sum_{1}^n$ či $\bigcup_{A \in B}$. V~případě vzorce $\lim\limits_{x\to 0}\frac{\sin x}{1} = 1$ jsme si vynutili méně úspornou sazbu příkazem \verb|\limits|.


\begin{eqnarray}
\int\limits_a^b f(x)\,\mathrm{d}x &= &-\int_b^a f(x)\,\mathrm{d}x\\
\Big( \sqrt[5]{x^4} \Big)' = \Big( x^{\frac{4}{5}} \Big)' &= & \frac{4}{5}x^{-\frac{1}{5}} = \frac{4}{5\sqrt[5]{x}}\\
\overline{\overline{A \vee B }} &= &\overline{\overline{A} \wedge \overline{B}} 
\end{eqnarray}

\section{Matice}
Pro sázení matic se velmi často používá prostředí \verb|array| a~závorky \verb|(\left,\right)|.

$$ \left( \begin{array}{cc}
a+b & b - a \\
\widehat{\xi + \omega} & \hat\pi\\
\vec{a} & \overleftrightarrow{AC}\\
0 & \beta
\end{array} \right) $$ 
 

$$ \mathbf{A} = \left\|\begin{array}{cccc}
a_{11} & a_{12} & \ldots & a_{1n}\\
a_{21} & a_{22} & \ldots & a_{2n}\\
\vdots & \vdots & \ddots & \vdots\\
a_{m1} & a_{m2} & \ldots & a_{mn}
\end{array}\right\| $$

$$ \left|\begin{array}{cc}
t & u\\ 
v & w
\end{array}\right| = tw - uv $$

Prostředí \verb|array| lze úspěšně využít i~jinde.
$$ \binom{n}{k} = \begin{cases}
\ \frac{n!}{k!(n-k)!} & \text{pro } 0 \leq k \leq n \\
\ 0 & \text{pro } k < 0 \text{ nebo } k > n \end{cases}$$

\section{Závěrem}
V případě, že budete potřebovat vyjádřit matematickou konstrukci nebo symbol a~nebude se Vám dařit jej nalézt v~samotném \LaTeX u, doporučuji prostudovat možnosti balíku maker \AmS -\LaTeX. 
   Analogická poučka platí obecně pro jakoukoli konstrukci v~\TeX u.


\end{document}